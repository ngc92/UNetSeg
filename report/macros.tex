\usepackage{xparse}

% math ops
\newcommand{\td}[1]{\mathrm{d}#1}
\newcommand{\dt}{\mathrm{d}t}
\newcommand{\pdiv}[2]{\frac{\partial #1}{\partial #2}}
\newcommand{\tdiv}[2]{\frac{\td{#1}}{\td{#2}}}
\newcommand{\definemapping}[3]{#1\colon #2 \ \longrightarrow\  #3\,}
\newcommand{\ix}[1]{_{\mathrm{#1}}}

\DeclareMathOperator{\const}{const}
\DeclareMathOperator{\sign}{sg}
\DeclareMathOperator{\argmin}{argmin}
\DeclareMathOperator{\argmax}{argmax}
\newcommand{\transpose}{^{\mathrm{T}}}

\newcommand{\uniform}[1]{\mathcal{U}\left( #1 \right)}
\newcommand{\mat}[1]{\mathbf{#1}}

\NewDocumentCommand{\functional}{mmo}{%
  #1 %
  \IfValueTF{#3}{_{#3}}{}%
  \!\left[ #2 \right]%
}

\NewDocumentCommand{\expectation}{om}{\functional{\mathbb{E}}{#2}[#1]}
\NewDocumentCommand{\probability}{om}{\functional{\mathbb{P}}{#2}[#1]}
\NewDocumentCommand{\variance}{om}{\functional{\mathbb{V}}{#2}[#1]}


\newcommand{\reals}{\mathbb{R}}
\newcommand{\pihalf}{\frac{\pi}{2}}
\newcommand{\sigmaalg}{\mathcal{F}}
\newcommand{\borel}{\mathcal{B}}
