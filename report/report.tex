%!TEX program = lualatex
\documentclass[aps,prl,twocolumn,groupedaddress,amsmath,amssymb]{revtex4-1}

% Math and stuff
\usepackage{amsmath,amssymb,amstext}
\usepackage{mathtools}
\usepackage{bbm} % indicator function \mathbbm{1}
\usepackage[no-math]{fontspec}
\setmainfont[Ligatures={TeX}]{Linux Libertine O}
\usepackage[libertine, cmintegrals]{newtxmath}
\usepackage{microtype}
\usepackage{amsfonts}

% Language, right quotes
\usepackage{polyglossia}
\setmainlanguage{english}
\usepackage{csquotes}



% Page geometry
\usepackage{geometry}

%% Images, graphics, drawings
%\usepackage{pgfplots}
\usepackage{tikz}
\usetikzlibrary{patterns,positioning,calc}
\usepackage{xcolor}
\usepackage[font=small,format=plain,labelfont=bf,up,textfont=it,up]{caption}
\usepackage{subcaption}
\usepackage{graphicx}

\usepackage{todonotes}
\usepackage{siunitx}

% Citing and linking
%\usepackage[style=alphabetic,backend=bibtex]{biblatex}
\usepackage{hyperref}

\usepackage{xparse}

% math ops
\newcommand{\td}[1]{\mathrm{d}#1}
\newcommand{\dt}{\mathrm{d}t}
\newcommand{\pdiv}[2]{\frac{\partial #1}{\partial #2}}
\newcommand{\tdiv}[2]{\frac{\td{#1}}{\td{#2}}}
\newcommand{\definemapping}[3]{#1\colon #2 \ \longrightarrow\  #3\,}
\newcommand{\ix}[1]{_{\mathrm{#1}}}

\DeclareMathOperator{\const}{const}
\DeclareMathOperator{\sign}{sg}
\DeclareMathOperator{\argmin}{argmin}
\DeclareMathOperator{\argmax}{argmax}
\newcommand{\transpose}{^{\mathrm{T}}}

\newcommand{\uniform}[1]{\mathcal{U}\left( #1 \right)}
\newcommand{\mat}[1]{\mathbf{#1}}

\NewDocumentCommand{\functional}{mmo}{%
  #1 %
  \IfValueTF{#3}{_{#3}}{}%
  \!\left[ #2 \right]%
}

\NewDocumentCommand{\expectation}{om}{\functional{\mathbb{E}}{#2}[#1]}
\NewDocumentCommand{\probability}{om}{\functional{\mathbb{P}}{#2}[#1]}
\NewDocumentCommand{\variance}{om}{\functional{\mathbb{V}}{#2}[#1]}


\newcommand{\reals}{\mathbb{R}}
\newcommand{\pihalf}{\frac{\pi}{2}}
\newcommand{\sigmaalg}{\mathcal{F}}
\newcommand{\borel}{\mathcal{B}}


\begin{document}

    \title{Supplementing UNets with a GAN Loss for Cell Segmentation}
    \author{Erik Schultheis}
    \maketitle

	\section{Introduction}

    In this work a deep network called a UNet \cite{ronneberger2015u} is used to generate
    segmentations from cell images. This means that given a black and white microscopy image of a
    slice of cells, we want the  network to detect the cell boundaries even if the quality of the
    original image is very low. To improve upon that architecture a regularizing loss inspired by
    generative adversarial  networks is added.

    The UNet architecture was chosen due to its success for other bioimaging segmentationt tasks. 
    It allows for precise localization of the segmentation classes and can be trained from few
    training examples. It is fully convolutional so it can be applied independent of the input
    image size. However, even the production of the low number (order of hundreds) of labelled
    training data can be expensive. 

    An alternative than does not need corresponding training data can be constructed with 
    generative adversarial networks. In this case a \emph{Generator} network constructs fake data,
    and a \emph{Discriminator} networks tries to distinguish it from real data. The competing 
    processes between genrator and discriminator result in the generator generating more and more
    realistic data. 

    By changing the generator to an image to image network, and introducing a cyclic consistency
    loss, this architecture can be used for image-to-image transfers as done e.g. in
    \emph{CycleGANs} \cite{zhu2017unpaired} and \emph{DiscoGANs} \cite{kim2017learning}.  The
    training requires data from both domains, but there is no need to have corresponding samples. In
    particular, one can use a data  set that contains cell images for which no corresponding ground-
    truth segmentation is available.
    
    Unfortunatle, the training process of GAN networks is far more unstable and slow than that of 
    UNets. For this reason we try to combine these two processes by adding a GAN-like loss as an
    additional regularization to the UNet. 

    In the following we first build the up UNet training process, taking a look at the effects
    of data augmentation, the choice of loss function, and the UNet architecture.

    \section{UNet}
    In this section the basic UNet model upon which the discriminator loss extension will be based,
    is developed. Several experiments are performed to find good settings for data augmentation, the
    loss function and the network architecture. Due to the high amount of hyperparameters, a 
    comparison of all possible settings is not possible. Instead, one aspect is varied and optimized
    while the others are kept fixed.

    \subsection{Data augmentation}
    Since there are only very few cell images and corresponding segmentations avaiable it is 
    necessary to make heavy use of data augmentation to facilitate training and generalization 
    of the network. The training data consists of 118 pairs of images, the evaluation data contains
    28 image pairs.

    The original cell images are given as 512 by 512 grayscale pixels. To reduce memory consumption
    these images are randomly cropped to $256\times256$. To increase the training set the images are
    randomly flipped (left-right and up-down) and rotated by multiples of 90 degrees. To further
    increase the variability in the  images the contrast is reduced by a random factor in the range
    $[0.25, 1]$. These images are then downsampled to $128\times128$ to reduce the class imbalance.
    For cell images this corresponds to a $2\times2$ average pooling, whereas  the binary
    segmentation images are resized with max pooling.

    The positions of the cell boundaries may not be specified pixel-precise by the cell images. To 
    penalize shifts of the order of a few pixels not as much as missing boundary classification, one 
    can replace to hard, binary classification ground truth with a slightly blurred version (
    Gaussian filter with $\sigma=1$). To ensure that the central portion of the cell boundary still
    gets classified as $100\%$ cell boundary, the original ground truth is added to the blurred 
    version, and clipped at $1$.

    An even stronger increase in image variability can be archieved when the changes in contrast are
    not applied homogenously to the image (in which case they can mostly be reverted by an  image
    standartization), but with a locally varying strength. For this a 16 by 16 mask is created and
    upscaled to the full image size, blurred, and then used to interpolate between the image and a
    version whose contrast has been reduced to $10\%$ of the original value. The same approach can
    be used to apply local variations in image brightness. This can be seen in
    figure~\ref{fig:augmentimage}.

    The effectiveness of this augmentation is demonstrated in figure~\ref{fig:augment}. Note that we
    cannot directly compare the effects on the  evaluation loss, as the blurring of segmentations
    affects the loss calculation.

    \begin{figure}[tbp]
        \begin{subfigure}[c]{0.45\linewidth}
            \includegraphics[width=\linewidth]{data/original.png}
            \caption{original}
        \end{subfigure}
        \begin{subfigure}[c]{0.45\linewidth}
            \includegraphics[width=\linewidth]{data/augmented.png}
            \caption{preprocessed}
        \end{subfigure}
        \caption{Result of applying the full preprocessing procedure to a cell image. 
        Notice the brighter and darker patches caused by local contrast and brigthness changes.}
        \label{fig:augmentimage}
    \end{figure}

    \begin{figure}[tbp]
        \begin{center}
        \includegraphics[width=\linewidth]{figures/augment.pdf}
        \end{center}
        \caption{Effects of more extensive data preprocessing (additional blurring and local 
        contrast/brightness variations) vs only simple (crop, flip, rotate) augmentations on the 
        IoU of the generated segmentations over the learning process.}
        \label{fig:augment}
    \end{figure}

    \subsection{Loss function}
    A second design decision for the training process is the choice of the loss function. 
    Interpreting the segmentations as probability maps for a pixel being on a cell boundary frames
    this task as a pixel classification problem and leads to a pixelwise binary (sigmoid) cross 
    entropy loss. Thinking of the problem as an image-to-image transfer would suggest a pixelwise
    mean squared error as a simple loss function. 

    In both cases the strong class imbalance constitutes a problem: Most pixels in the target 
    segmentations are black, so a network that outputs all-black images would already get a high 
    accuracy. This can be countered by giving more weight to white pixels, which is automatically 
    done in the following way: The percentage of white $\bar{w}$ in the image is calculated 
    and the loss at a pixel of value $p$ is weighted as $w = p/\bar{w} + (1-p)/(1-\bar{w})$. If there
    are 9 times more black pixels than white, $\bar{w}=0.1$ and white pixels are weighted 9 times 
    stronger than black pixels.

    As demonstrated in figure~\ref{fig:mse} the cross-entropy based training seems to slightly, but
    consistently, outperform the mean-squared-error approach. However, it must be noted that the 
    tests where performs with the same learning rate, so fine-tuning for the speicific loss function 
    might be able to improve the results. Still, we takes this as enough of an indication to use the
    cross-entropy loss for all further considerations.

    \begin{figure}[tbp]
        \begin{center}
        \includegraphics[width=\linewidth]{figures/mse.pdf}
        \end{center}
        \caption{Comparison of learning progress using the cross entropy and mean squared error 
        losses. White pixels (cell boundaries) are weighted more strongly than black pixels. 
        Cross-entropy consistently exceeds MSE in performance.}
        \label{fig:mse}
    \end{figure}

    \subsection{Architecture}
    The Unet architecture consists of resolution decreasing downward blocks, followed by resolution
    increasing upwards blocks. To improve the spatial precision, each upward block gets  as input
    both the result of the preceding block, and that of the corresponding downward block at the same
    resolution. As a downward block two convolutions with $3\times3$ receptive field are followed by
    a $2\times2$ max-pooling. With each deeper layer the  number of feature maps doubles. The
    convolutions are padded such that the resulting feature map has the same resolution as the
    original one.

    When increasing the spatial size of a feature layer by deconvolutions, the network is
    prone to generate so-called checkboard artifacts \cite{odena2016deconvolution}. While the 
    supervised nature of the UNet training suppressed these problems, we still investigate whether
    performance can be improved by using a model inherently resistent to these effects. For that
    the deconvolutions are replaced by an upscaling, followed by a regular convolution. 

    The effects of different network depths (measured as the number of downward convolution blocks),
    and the choice of upscaling instead of deconvolution, are shown in figure~\ref{fig:arch}. As the
    results are only marginally different, the depth-4 network is chosen, as it may have additional,
    unused modelling capacity that could become useful when the additional loss is introduced below.

    \begin{figure}[tbp]
        \begin{subfigure}[c]{\linewidth}
            \includegraphics[width=\linewidth]{figures/arch.pdf}
            \caption{IoU}
        \end{subfigure}
        \begin{subfigure}[c]{\linewidth}
            \includegraphics[width=\linewidth]{figures/arch_xent.pdf}
            \caption{Cross-entropy}
        \end{subfigure}
        \caption{Comparison of different network architectures in terms of segmentation IoU and
        cross entropy. It turns out that the precise choice of network architecture does not 
        have a significant influence on the performance.}
        \label{fig:arch}
    \end{figure}

    \section{Regularisation by Discriminator}

    \begin{figure}[tbp]
        \begin{center}
        %\tikzstyle{neuron} = [draw, circle, text centered, text width=0.75cm, minimum height=0.5cm, fill=white]

\pgfdeclarelayer{background}
\pgfdeclarelayer{foreground}
\pgfsetlayers{background,foreground}

\begin{tikzpicture}[node distance=1.5cm, auto]
    \begin{pgfonlayer}{foreground}
        \node (A) {$A$};
        \node [below of=A] (B) {$B$};
        \node [right of=A] (unet) {UNet};
        \node [right of=unet] (fake) {$B^{\prime}$};

        \node [right of=fake, circle, fill=white, yshift=-0.8cm] (disc) {D};
        \node [right of=disc, yshift=0.6cm, xshift=-0.5cm] (fakec) {$d_{\mathrm{g}}$};
        
        \node [right of=disc, yshift=-0.6cm, xshift=-0.5cm] (realc) {$d_{\mathrm{r}}$};
        \node [right of=realc, xshift=-0.5cm] (Ldr) {$L_{\mathrm{d, r}}$};

        \node [above of=fakec] (Lgc) {$L_{\mathrm{g, c}}$};
        \node [right of=fakec, xshift=-0.5cm] (Ldf) {$L_{\mathrm{d, f}}$};

        \node [draw, circle, inner sep=1.0pt] (Ld) at ($(Ldr)!0.5!(Ldf)$) {$+$};

        \node [above right=0.4cm and 0.0cm of fake.east] (xent) {$L_{\mathrm{U}}$};
        \node at (xent|-Lgc) [draw, circle, fill=white, inner sep=1.0pt] (Lg) {$+$};

    \end{pgfonlayer}

    \begin{pgfonlayer}{background}
        % Data flow: UNet path
        \draw[->] (A) -- (unet) -- (fake);
        \draw[->] (fake) -- (fake|-xent) -- (xent);
        \draw[->] (xent|-B) -- (xent);
        \node at (xent|-B) [circle,fill,inner sep=1.0pt]{};
        \draw[->] (xent) -- (Lg);

        % Data flow: Generator path
        \draw[->] (fake) -- ([xshift = -0.5cm]disc.center|-fake);
        \draw[->] ([xshift=0.4cm]disc.center) -- (fakec);
        \draw[->] (fakec) -- (Lgc);
        \draw[->] (fakec) -- (Ldf);

        % Data flow: Discriminator path
        \draw[->] (B) -- ([xshift = -0.5cm]disc.center|-B);
        \draw[->] ([xshift=0.4cm]disc.center) -- (realc);
        \draw[->] (realc) -- (Ldr);
        \draw[->] (Ldr) -- (Ld);
        \draw[->] (Ldf) -- (Ld);
        

        % The UNet Symbol
        \draw[fill=white] (unet) -- ++ (-1, -0.9) -- ++ (0, 1.8) -- (unet);
        \draw[fill=white] (unet) -- ++ (+1, -0.9) -- ++ (0, 1.8) -- (unet);

        \draw [->] (unet) ++ (-0.5, +0.5) -- ++ (1, 0);
        \draw [->] (unet) ++ (-0.5, -0.5) -- ++ (1, 0);

        % The discriminators
        \draw[fill=white] (disc) ++ (+0.5, 0) -- ++ (-1, -1.0) -- ++ (0, 2.0) -- cycle;
       
        \draw (Ld.east) -- ++ (0.2, 0) -- ++ (0, -1.1) -- ++ (-1.8, 0);
        \draw[->, thick] (Ld.east) ++ (-1.6, -1.1) -- ([xshift=-0.7cm, yshift=0.3cm]disc.center);
    \end{pgfonlayer}

    % The update lines need to be in the foreground
    \begin{pgfonlayer}{foreground}
        \draw (Lgc.west) -- ([xshift=0.5cm]unet|-Lgc.west);
        \draw[->, thick] ([xshift=0.5cm]unet|-Lgc.west) -- ++ (-1.7, -1.7);

        % for some reason this needs to be redrawn here!
        \node at (xent|-Lgc) [draw, circle, fill=white, inner sep=1.0pt] (Lg) {$+$};
    \end{pgfonlayer}

\end{tikzpicture}
        \end{center}
        \caption{Architecture of the UNet. Bold arrows indicate weight updates by gradient descent. 
        The UNet loss $L\ix{U}$ is based on a (pixel wise) comparison of $B$ and $B^{\prime}$. The
        discrimination losses are based on the discrimination results $d\ix{r} = D(B)$, 
        $d\ix{g}=D(B^{\prime})$. They are $L\ix{g, c} = \mathrm{xent}(d\ix{g}, 1)$, 
        $L\ix{d, f} = \mathrm{xent}(d\ix{g}, 0)$ and $L\ix{d, r} = \mathrm{xent}(d\ix{r}, 1)$.}
        \label{fig:unet_dics}
    \end{figure}
    
    The UNet architcture above is now supplemented with a discriminator network to further refine
    the generated segmentations as shown in figure~\ref{fig:unet_dics}. The disciminator follows the
    architecture used in DCGANs \cite{dcgan}, i.e. it consists of blocks of a convolution followed
    by BatchNorm followed by a leaky ReLU. Since the UNet is fully convolutional and thus can act on
    input images of arbitrary size, it made sense to also construct a fully convolutional
    discriminator, which is achieved by the last layer being a global average pooling into a single
    unit.

    Unfortunately the discriminator training is very unstable. This can be seen in
    figure~\ref{fig:gan_adam}. While initially better the UNet+GAN combination at about $5000$
    steps. This corresponds to a decrease in the discriminator loss and an \emph{increase} in the
    generator loss -- it appears that the discriminator becomes to successfull. Replacing its 
    optimizer with the much slower SGD (with momentum $0.9$) does precent the instability, but also
    no longer shows the improved performance in the early stages of learning. 

    Even with 
    significantly increased learning rate ($1e-4$ instead of $5e-5$) the discriminator remains  

    \begin{figure}[tbp]
        \begin{center}
        \includegraphics[width=\linewidth]{figures/disc_sgd_adam.pdf}
        \end{center}
        \caption{Adding a GAN loss makes the learning start out better, but leads to instability. 
        Using a much slower SGD discriminator prevents this from happening, but also no longer 
        improves performance.}
        \label{fig:gan_adam}
    \end{figure}

    \begin{figure}[tbp]
        \begin{center}
        \includegraphics[width=\linewidth]{figures/dloss_sgd_adam.pdf}
        \end{center}
        \caption{Comparison of discriminator losses. SGD does not result in a significant
        reduction of discriminator loss, making the larning stable but also preventing the GAN from
        providing a helpful training signal.}
        \label{fig:dloss}
    \end{figure}

    The loss function for the generator consists of a cross entropy loss that makes the generator
    try to fool the discriminator, and a feature matching \cite{salimans2016improved} loss that
    tries to ensure that the  features in the hidden layers of the discriminator follow the same
    statics (in this case mean), independently of whether the input comes from a real segmentation
    of from a generated one. In this case it does not seem to have significant influence on the 
    learning performance, see figure~\ref{fig:gan_fm}.

    \begin{figure}[tbp]
        \begin{center}
        \includegraphics[width=\linewidth]{data/fm.pdf}
        \end{center}
        \caption{Comparison of learning progress with and without feature matching.}
        \label{fig:gan_fm}
    \end{figure}

    Furthermore, since we want to train the discriminator on images that look somewhat realistic, 
    we only start training it after having trained the UNet for 500 steps. The generator loss, in 
    turn, is only sensible for a sensible discriminator, so this additional loss is only enabled 
    after 750 steps. 

    As for the UNet part, there are hyperparameters that need to be tuned. One parameter that turned
    out to be very important is the learning rate for the discriminator parameters. If this is set 
    too high, the discriminator becomes good quickly and the generator performance deteriorates.
    It was chosen to be $5\times10^{-4}$ as this value results in the discriminator loss remaining 
    approximately constant over the learning process.

    A second parameter is given by the strength of the GAN loss relative to the UNet classification
    loss. Therefore the generator loss is scaled by an additional parameter lambda. The training 
    progress for different values of $\lambda$ can be seen in figure~\ref{fig:gan_strength}. For low
    enough $\lambda$ the gan based approach outperforms the simple UNet. However, the training 
    becomes more unstable as can be seen be the increased variations in the learning curves.

    \begin{figure}[tbp]
        \begin{center}
        \includegraphics[width=\linewidth]{data/gan_strength.pdf}
        \end{center}
        \caption{Comparison of learning progress using different scaling factors for the GAN loss.
        Having a too strong contribution from the gan makes the training unstable and 
        deteriorates performance (yellow).}
        \label{fig:gan_strength}
    \end{figure}

    Taking a look at the generated images fig.~\ref{fig:unet_vs_ganunet} reveals another problem: 
    The GAN loss seems to cause severe checkerboard artifacts in the generated image. Replacing the
    deconvolutions in the UNet with upscaling layers fixes this problem. In this case, however, 
    using $\lambda=0.25$ makes the network too sensitive. At $\lambda=0.1$ is produces segmentations
    that have sharper lines than in the pure UNet approach.

    \begin{figure}[tbp]
        \begin{subfigure}[c]{0.45\linewidth}
            \includegraphics[width=\linewidth]{data/27_orig.png}
            \caption{Cell image}
        \end{subfigure} %
        \begin{subfigure}[c]{0.45\linewidth}
            \includegraphics[width=\linewidth]{data/27_seg.png}
            \caption{Segmentation}
        \end{subfigure} \\
        \begin{subfigure}[c]{0.45\linewidth}
            \includegraphics[width=\linewidth]{data/27_simple.png}
            \caption{$\lambda=0$}
        \end{subfigure}%
        \begin{subfigure}[c]{0.45\linewidth}
            \includegraphics[width=\linewidth]{data/27_0_1.png}
            \caption{$\lambda=0.1$}
        \end{subfigure}\\
        \begin{subfigure}[c]{0.45\linewidth}
            \includegraphics[width=\linewidth]{data/27_up_0_1.png}
            \caption{upscaling, $\lambda=0.1$}
        \end{subfigure}%
        \begin{subfigure}[c]{0.45\linewidth}
            \includegraphics[width=\linewidth]{data/27_up_0_25.png}
            \caption{upscaling, $\lambda=0.25$}
        \end{subfigure}
        \caption{Comparison of resulting image for the UNet and UNet with GAN loss.
         The additional discrimination loss causes the segmented cell boundaries
        to become more crisp, but at the outer regions produces significant checkerboard artifacts.
        This can be countered by using an upscaling network (last row).
        Note that the networks detect more lines than are present in the ground truth data.}
        \label{fig:unet_vs_ganunet}
    \end{figure}

    Looking at the resulting learning curves figure~\ref{fig:gan_up} shows that the advantage 
    (in IoU) over pure UNet is smaller than in the deconvolution approach. In particular the 
    $\lambda=0.25$ case reaches a point where there is a hugh performance drop (that might have been
    compensated again if the training had continued).

    \begin{figure}[tbp]
        \begin{center}
        \includegraphics[width=\linewidth]{data/gan_up.pdf}
        \end{center}
        \caption{IoU over learning steps for different GAN loss weightings $\lambda$ in the
        case that the UNet uses upscaling instead of deconvolutions. The gain in IoU is lower
        and higher $\lambda$ leads to deteriorating performance, but the resulting images are
        visually improved.}
        \label{fig:gan_up}
    \end{figure}

    A look into the workings of the GAN is provided in figure~\ref{fig:gan_losses}. The 
    discriminator loss decreases slowly whereas the generator loss increases, but still remains 
    significantly under the boundary of $1$ at which it would be clipped. However, this behaviour 
    suggests that letting the training continue for much longer might reduce in very unstable 
    results.
    \begin{figure}[tbp]
        \begin{subfigure}[c]{\linewidth}
            \includegraphics[width=\linewidth]{data/gan_disc.pdf}
            \caption{Discriminator Loss}
        \end{subfigure}
        \begin{subfigure}[c]{\linewidth}
            \includegraphics[width=\linewidth]{data/gan_gen.pdf}
            \caption{Generator Loss}
        \end{subfigure}
        \caption{Losses of discriminator and generator over the learning progress. A strong weighting
        of the generator loss in the UNet learning makes it more difficult for the discriminator to 
        reduce its loss.}
        \label{fig:gan_losses}
    \end{figure}

    \section{Sample Efficiency}
    A first experiment in sample efficiency can be performed by reducing the training set to only 20
    images. This drastic resolution to about a sixth of the original training data is handled rather
    well by the UNet approach. While there is some overfitting and a reduction in IoU, the network
    still generalizes to the test set and produces reasonable results.

    Adding the GAN loss provides enough regularization to compensate for the lack of training data 
    -- in fact in sometimes even exceeds to regular UNet on the full dataset, see 
    figure~\ref{fig:reduced_data}.

    \begin{figure}[tbp]
        \begin{center}
        \includegraphics[width=\linewidth]{data/reduced_data.pdf}
        \end{center}
        \caption{IoU when training only on 20 examples, with and without GAN loss. For reference
        also the behaviour on the full training set is shown. In this scenario the additional GAN
        loss provides a significant improvement.}
        \label{fig:reduced_data}
    \end{figure}


    \section{Discussion and Further Work}
    A significant problem in training and evaluation the segmentations is due to the quality of the
    ground truth data. As the network becomes good it starts to (correctly) mark more cell 
    boundaries at the outer fringes of the cell image. However, these lines are not part of the 
    ground-truth data (because the quality of the original image in these regions is really low), so
    the cross-entropy loss will train the network to not detect these boundaries. See for example 
    the left side of the images in figure~\ref{fig:unet_vs_ganunet}.

    This could be alleviated by providing an additional map that contains information about the 
    condidence of the ground truth data -- so that the cross entropy could be weighted less in low
    confidence regions.

    For the evaluation procedure this means that it is impossible for the loss to become smaller 
    (or the IoU larger) than some threshold that is determined by the data quality.

    Another possible area of improvement is the data augmentation pipeline. The current 
    augmentations have the advantage of transforming the segmentations pixel precise. If one lifts 
    this restriction, arbitrary rotations and shears, as well as slight scale variations, could be
    added as preprocessing modifications.

    As above demonstrated, the GAN loss can be used to improve the training when there are only few
    training examples. Going even further, one could modify the training procedure to only require 
    partially labeled data. For any data point for which no ground truth segmentation exists, the 
    UNet should still generate a segmentation that the discriminator accepts as ``realistic''. 


    \section{Summary}

    A UNet can be used to detect cell boundaries in microscopy images. By aggressive data
    augmentation the training can be further improved. Introducing an additional discriminator loss
    that judges produced segmentations for whether the seem ``realistic'' can further improve the 
    networks performance, but requires some fine-tuning and makes the training process more 
    unstable. In a setting where labelled data is extremely scarce, the regularisation provided
    by the discriminator becomes more useful.




    \bibliographystyle{plain}
    \bibliography{lit.bib} 
\end{document}
